\documentclass[11pt]{exam}
%\usepackage[activeacute,spanish]{babel}
\usepackage[utf8]{inputenc}
\usepackage{amsmath,epsfig}
\usepackage[colorlinks]{hyperref}

\usepackage{minted} 
\usemintedstyle{emacs}
\usepackage{tcolorbox} % colores para el fondo
\definecolor{bg}{rgb}{0.95,0.95,0.95} %color de fondo

\pagestyle{headandfoot}
%\spanishdecimal{.}

\begin{document}

\firstpageheadrule
\runningheadrule
\lhead{\bf\normalsize Taller Python 2019}
\rhead{\bf\normalsize Gu\'ia M\'odulos y Clases}
\cfoot{ }
\lfoot{\tiny SV/GR}
\begin{flushleft}
\vspace{0.2in}

\vspace{0.25cm}
\end{flushleft}
%%%%%%%%%%%%%%%%%%%%%%%%%%%%%%%%%%%%%%%%%%

\begin{questions}
\item Copie los códigos que escribió anteriormente y que definen su implementación de  las funciones \texttt{primos(n)} y \texttt{mifactorial(n)}, y además el código que calcula de su aproximación de $\pi$ (almacenada ahora en la variable \texttt{mipi}) en un nuevo archivo llamado \texttt{misfunciones.py}. Este archivo puede usarse para definir un nuevo módulo. A continuación, en una sesión interactiva de Python, importe su función usando primero \texttt{import mimodulo} y llame a las funciones que están ahí definidas. A continación, importe el módulo de la segunda forma vista en clases, es decir, usando \texttt{from mimodulo import *} y vea cómo funciona ahora la llamada a las funciones.

\item El factorial es una función comúnmente usada, y ya está implementada en diversos módulos populares de Python, por ejemplo, en el módulo \texttt{math}. Para verificar esto, importe el módulo \texttt{math} de las tres formas discutidas en clase y verifique que la función \texttt{factorial} entrega los mismos valores ya calculados por usted. Lo mismo ocurre con el valor del número $\pi$ (\texttt{math.pi}). 

\item Aproveche que tiene cargado el módulo \texttt{math} e investigue qué funciones y variables están definidas en este módulo. Para esto, ejecute \texttt{dir(math)} o revise la \href{https://docs.python.org/2/library/math.html}{documentaci\'on en l\'inea} disponible.

\item Construya una clase que salude a una persona de dos formas, 
      una con \textit{Mayúsculas} y otra con \textit{Minúsculas} (la puede nombrar \texttt{Saludos}). 
      Inicie la clase con el nombre de la persona. 

\item Construya una clase para la forma de un rectángulo. Inicie la 
      clase con la distancia $x$ e $y$ del objeto abstracto. Además incluya un método
      que calcule el área, perímetero del rectángulo y uno que revise si es un cuadrado.

\item Construya una clase para un puente. Inicie la clase con los siguientes atributos:
        \begin{itemize}
            \item Largo del puente.
            \item Tipo del puente.
            \item Alturo del puente.
            \item Material.
        \end{itemize}
     En base a esto escriba dos métodos, uno que le asigne un peso máximo al puente y otro
     que calcule cuántos vehículos puede soportar el puente en base al peso máximo. (Hint: Peso de un vehículo promedio: 1500 kg) (Avanzado!)

\item Importe el módulo \texttt{numpy} con el nombre \texttt{np} e imprima todas las funciones, variables, y clases que contiene. A continuación, importe solamente la función \texttt{array} y la clase \texttt{ndarray} desde \texttt{numpy}.

%\item Cree un módulo Python con todas las clases y funciones que ha escrito hasta este momento. Éste tiene que ser un archivo separado, que se pueda llamar desde cualquier programa.
\end{questions}
\end{document}
