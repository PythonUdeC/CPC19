\documentclass[11pt]{exam}
%\usepackage[activeacute,spanish]{babel}
\usepackage[utf8]{inputenc}
\usepackage{amsmath,epsfig}
\usepackage[colorlinks]{hyperref}

\usepackage{minted} 
\usemintedstyle{emacs}
\usepackage{tcolorbox} % colores para el fondo
\definecolor{bg}{rgb}{0.95,0.95,0.95} %color de fondo

\pagestyle{headandfoot}
%\spanishdecimal{.}

\begin{document}

\firstpageheadrule
\runningheadrule
\lhead{\bf\normalsize Taller Python 2019}
\rhead{\bf\normalsize Gu\'ia M\'odulos y Clases}
\cfoot{ }
\lfoot{\tiny GR}
\begin{flushleft}
\vspace{0.2in}

\vspace{0.25cm}
\end{flushleft}
%%%%%%%%%%%%%%%%%%%%%%%%%%%%%%%%%%%%%%%%%%

\begin{questions}
%\item Copie los c'odigos que escribi'o anteriormente y que definen su implementación de  las funciones \texttt{primos(n)} y \texttt{mifactorial(n)}, y adem'as c'odigo que calcula de su aproximaci'on de pi (almacenada ahora en la variable \texttt{mipi}) en un nuevo archivo llamado \texttt{misfunciones.py}. Este archivo puede usarse para definir un nuevo m'odulo. A continuaci'on, en una sesi'on interactiva de Python, importe su funci'on usando primero \texttt{import mimodulo} y llame a las funciones que est'an ah'i definidas. A continaci'on, importe el módulo de la segunda forma vista en clases, es decir, usando \texttt{from mimodulo import *} y vea c'omo funciona ahora la llamada a las funciones.
%
%\item El factorial es una funci'on com'unmente usada, y ya est'a implementada en diversos m'odulos populares de Python, por ejemplo, en el m'odulo \texttt{math}. Para verificar esto, importe el m'odulo \texttt{math} de las tres formas discutidas en clase y verifique que la funci'on \texttt{factorial} entrega los mismos valores ya calculados por usted. Lo mismo ocurre con el valor del n'umero $\pi$ (\texttt{math.pi}). 
%
%\item Aproveche que tiene cargado el m'odulo \texttt{math} e investigue qu'e funciones y variables est'an definidas en este m'odulo. Para esto, ejecute \texttt{dir(math)} o revise la \href{https://docs.python.org/2/library/math.html}{documentaci\'on en l\'inea} disponible.
%\end{questions}

\item Construya una clase que saludo a una persona de dos formas, 
      uno con MAYUSCULAS y otro con MINUSCULAS. (la puede nombrar Saludos). 
      Inicie la clase con el nombre de la persona. 

\item Construya una clase para la forma de un rectangulo. Inicie la 
      clase con la distancia $x$ e $y$ del objeto abstracto. Además incluya un método
      que calcula el área, perimetero del rectangulo y uno que revisa si es un cuadrado.

\item Construya una clase para un puente. Inicie la clase con los siguientes atributos:
        \begin{itemize}
            \item Largo del puente
            \item Tipo del puente
            \item Alturo del puente
            \item Material
        \end{itemize}
     En base a eso escriba dos métodos, uno que le asigna un peso máximo al puente y otro
     que calcula cuantos vehiculos puede soportar el puente en base al peso máximo. (Hint: Peso 
     de un vehiculo promedio: 1500 kg) (Avanzado!)

\item Importe el módulo de numpy con el nombre np y imprima todas las funciones, variables,
      y clases que contiene. Despues importe solamente la función array y la clase ndarray desde
      numpy.

\item Haga un modulo Python con todas las clases y funciones que ha escrito hasta este momento.
      Este tiene que ser un archivo separado, que se pueda llamar desde cualquier programa.
\end{questions}
\end{document}
