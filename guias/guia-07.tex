\documentclass[11pt]{exam}
\usepackage[activeacute,spanish]{babel} % Permite el idioma espa\~nol.
\usepackage[latin1]{inputenc}
\usepackage{amsmath,amsfonts}
\usepackage[colorlinks]{hyperref}
\usepackage{graphicx}
\usepackage{hyperref}

%added because of problem between framed and exam package
\newcommand*{\renameenviron}[1]{%
  \expandafter\let\csname exam-#1\expandafter\endcsname
      \csname #1\endcsname
  \expandafter\let\csname endexam-#1\expandafter\endcsname
      \csname end#1\endcsname
  \expandafter\let\csname #1\endcsname\relax
  \expandafter\let\csname end#1\endcsname\relax
}
\renameenviron{framed}
\renameenviron{shaded}
\renameenviron{leftbar}
\usepackage{framed}

\usepackage{minted} 

\pagestyle{headandfoot}
\spanishdecimal{.}

\begin{document}

\firstpageheadrule
%\firstpagefootrule
%\firstpagefooter{}{Pagina \thepage\ de \pages}{}
\runningheadrule
%\runningfootrule
\lhead{\bf\normalsize Taller Python 2018}
\rhead{\bf\normalsize Gu\'ia 07}
\cfoot{ }
\lfoot{\tiny EVM}
\begin{flushleft}
\vspace{0.2in}
%\hbox to \textwidth{Nombre: \enspace \hrulefill}
%Nombre : \\
\vspace{0.25cm}
\end{flushleft}
%%%%%%%%%%%%%%%%%%%%%%%%%%%%%%%%%%%%%%%%%%

En esta pr'actica usted ejercitar'a el procesamiento de archivos usando las funciones propias de python, las m'as avanzadas de \texttt{numpy}, y \texttt{pandas} para tratar bases de datos. Los archivos que requiere para la gu'ia lo encontrar'a en el directorio \texttt{Data$/$Guia} de la unidad.
\begin{questions}



\item Use las funciones de python (sin utilizar numpy) para leer el archivo "visitantes\_cine\_2016.dat". Defina correctamente la ruta al archivo con la variable \texttt{str\_file} y use las tres funciones descritas abajo.

\begin{minted}{python}
# -*- coding: UTF-8 -*-

str_file = '[Reemplazar con ruta al archivo visitantes_cine_2016.dat]'


f_in = open(str_file,'r')

%ejemplo de read
data = f.read

%ejemplo de readline
data = f.readline

%ejemplo de realines
data = f.readlines
\end{minted}
No olvide abrir y cerrar el archivo despu'es de procesarlo con las funciones \texttt{open} y \texttt{close}. Compare los distintos tipos y contenidos de la variable \texttt{data} como resultado de cada funci'on.

\item Lea el primer rengl'on del mismo archivo y asigne el contenido a dos variables, una con nombre \texttt{mes} que contenga el mes y la otra \texttt{vis} con el n'umero de visitantes.

\item Procese todo el archivo con un bucle (loop) y guarde el contenido en dos listas con el mismo nombre de las variables del ejercicio anterior.

\item Grafique con matplotlib el n'umero de visitantes en funci'on del mes. Calcule el n'umero de visitantes anuales y el promedio de visitantes mensuales.

\item Escriba en un nuevo archivo el n'umero de visitantes anuales y el promedio de visitantes mensuales en dos lineas diferentes.

\item Use la funci'on \texttt{genfromtxt} o \texttt{loadtxt} del m'odulo \texttt{numpy} para procesar el archivo \texttt{msd.xvg} que se encuentra en el directorio \texttt{Data$/$Guia}. Este archivo contiene la desviaci'on quadr'atica media de un part'icula (CO$_{2}$ en agua) en $nm^2$ en funci'on del tiempo de simulaci'on en $ps$. Grafique estas dos variables y realice una regresi'on lineal para calcular el coeficiente de difusi'on que equivale a un sexto de la pendiente de la regresi'on. (El valor obtenido por otro programa lo encuentra como un comentario al inicio del archivo msd.xvg)
	
\newpage	

Finalmente, vamos a trabajar con una base de datos del banco central. Bajo el siguiente link \url{http://si3.bcentral.cl/Siete/secure/cuadros/arboles.aspx} podr'a encontrar el valor del dolar observado en el a\~{n}o 2016.
\item Gaurde los valores del dolar con formato excel
\item Abra el archivo descargado con Microsoft excel y gaurde el archivo en formato csv (en caso de no tener Microsoft excel puede usar el archivo que se encuentra bajo el directorio \texttt{Data$/$Guia}.) 
\item Procese el archivo con la funci'on \texttt{genfromtxt} para obenter el valor promedio del dolar observado en el a\~{n}o 2016. (Para eliminar los valores ausentes de los d'ias feriados puede utilizar la funci'on \texttt{isnan()} de numpy.)

\item \textbf{Bonus Track (opcional)}: Busque en internet bajo el link \url{http://www.datos.gob.cl} una base de datos con las precipitaciones en Chile del primer semestre del a\~{n}o 2015. Descargue el archivo con formato \texttt{.csv}. Procese el archivo con el m'odulo pandas y analice los datos para responder a la siguientes preguntas: a. Cu'al estaci'on registro la mayor precipitaci'on diaria y en que fecha ocurrio esto? b. Cu'al fue la precipitaci'on acumulada en la estaci'on de Pichoy, Valdivia durante el primer semestre de este a\~{n}o? (Para esta 'ultima pregunta se recomienda usar la funci'on \texttt{loc} en \texttt{pandas})
\end{questions}
\end{document} 
