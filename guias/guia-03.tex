\documentclass[11pt]{exam}
\usepackage[activeacute,spanish]{babel} % Permite el idioma espa\~nol.
\usepackage[latin1]{inputenc}
\usepackage{amsmath,amsfonts}
\usepackage[colorlinks]{hyperref}
\usepackage{graphicx}
\usepackage{minted} 

\pagestyle{headandfoot}

\spanishdecimal{.}

\begin{document}

\firstpageheadrule
%\firstpagefootrule
%\firstpagefooter{}{Pagina \thepage\ de \pages}{}
\runningheadrule
%\runningfootrule
\lhead{\bf\normalsize Taller Python 2018}
\rhead{\bf\normalsize Gu\'ia 03}
\cfoot{ }
\lfoot{\tiny GR}
\begin{flushleft}
\vspace{0.2in}
%\hbox to \textwidth{Nombre: \enspace \hrulefill}
%Nombre : \\
\vspace{0.25cm}
\end{flushleft}
%%%%%%%%%%%%%%%%%%%%%%%%%%%%%%%%%%%%%%%%%%

\begin{questions}

\item Modifique el programa \texttt{test.py} que cre'o en la gu'ia 01 y que resuelve la ecuaci'on cuadr'atica $ax^2+bx+c=0$, para que ahora el programa informe que existen dos soluciones reales, y las imprima, si el discriminante $b^2-4ac$ es positivo, o que informe que no existe soluci'on real (si el discriminante es negativo), o bien que informe que existe s'olo una soluci'on real, y la imprima (si el discriminante es nulo).
\item Escriba un programa en Python que calcule el valor de la siguiente suma:
\begin{equation}
1+\frac{1}{2}+\frac{1}{2^2}+\frac{1}{2^3}+\cdots+\frac{1}{2^{99}}+\frac{1}{2^{100}}.
\end{equation}
\textbf{Advertencia!}: cuidado con la divisi'on por entero. Recuerde que en Python la divisi'on de dos enteros suministra la parte entera del cuociente (\texttt{1/2=0}). En este ejercicio, por otro lado, usted debe calcular un \texttt{float} (n'umero con decimales).

\item El factorial de un n'umero entero positivo $n$, denotado por $n!$ es definido por
\begin{equation}
n!=1\cdot 2\cdot 3\cdots (n-1)\cdot n.
\end{equation}
Por ejemplo, $3!=1\cdot 2\cdot 3=6$ y $10!=3628800$.
Escriba un programa en Python que pregunte al usuario por el valor de $n$, y que calcule e imprima su factorial, es decir, $n!$.
\item Modifique el c'odigo anterior para que su programa verifique, antes de calcular el factorial, que el n'umero suministrado es realmente un entero positivo, y s'olo calcule el factorial en ese caso, y que en caso contrario informe al usuario que el n'umero ingresado no es apropiado.

\item Modifique el programa anterior para que ahora el factorial defina una funci'on \texttt{mifactorial}, de modo que el factorial de $n$ se pueda luego llamar como \texttt{mifactorial(n)}.

\item Guarde el c'odigo anterior, que define su funci'on factorial, en un archivo \texttt{misfunciones.py}. Este archivo constituye un \textit{m'odulo}. A continuaci'on, en una sesi'on interactiva de Python, importe su funci'on usando primero \texttt{import misfunciones} y llame a su funci'on. Luego, importe ahora la funci'on de la segunda forma vista en clases, es decir, usando \texttt{from misfunciones import *} y vea c'omo funciona la llamada a la funci'on ahora.

\item El factorial es una funci'on com'unmente usada, y ya est'a implementada en diversos m'odulos populares de Python, por ejemplo, en el m'odulo \texttt{math}. Para verificar esto, importe el m'odulo \texttt{math} de las tres formas discutidas en clase\footnote{Ver el final de \href{http://nbviewer.ipython.org/github/gfrubi/clases-python-cientifico/blob/master/Lecture-1-Introduction-to-Python-Programming.ipynb}{estos apuntes} si no se acuerda lo discutido en clases.} y verifique que la funci'on \texttt{factorial} entrega los mismos valores ya calculados por usted. Aproveche que tiene cargado el m'odulo \texttt{math} e investigue\footnote{Ver por ejemplo la \href{https://docs.python.org/2/library/math.html}{documentaci\'on} en ingl'es de Python.} qu'e funciones y variables est'an definidas en este m'odulo.
\end{questions}

\end{document} 
