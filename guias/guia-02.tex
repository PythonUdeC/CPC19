\documentclass[11pt]{exam}
\usepackage[activeacute,spanish]{babel} % Permite el idioma espa\~nol.
\usepackage[latin1]{inputenc}
\usepackage{amsmath,amsfonts}
\usepackage[colorlinks]{hyperref}
\usepackage{graphicx}
\usepackage{minted} 

\pagestyle{headandfoot}

\spanishdecimal{.}

\begin{document}

\firstpageheadrule
%\firstpagefootrule
%\firstpagefooter{}{Pagina \thepage\ de \pages}{}
\runningheadrule
%\runningfootrule
\lhead{\bf\normalsize Taller Python 2018}
\rhead{\bf\normalsize Gu\'ia 02}
\cfoot{ }
\lfoot{\tiny GR}
\begin{flushleft}
\vspace{0.2in}
%\hbox to \textwidth{Nombre: \enspace \hrulefill}
%Nombre : \\
\vspace{0.25cm}
\end{flushleft}
%%%%%%%%%%%%%%%%%%%%%%%%%%%%%%%%%%%%%%%%%%

\begin{questions}

\item Los caracteres individuales que forman una cadena alfanum'erica (string) pueden ser accesados usando el n'umero del \textit{'indice} correspondiente. En Python \textbf{el valor de los 'indices siempre comienza en 0}, luego 1, 2, etc. Para ilustrar esto, abra un int'erprete Python (o IPython) y ejecute los siguientes comandos:
\begin{minted}{python}
x="Hola estudiantes de primer semestre"
print(x[0])
print(x[1])
print(x[2])
print(x[3])
\end{minted}
\item Como comprob'o en la pr'actica anterior la funci'on \texttt{len()} entrega el largo del string, es decir, el n'umero de caracteres que contiene. Por lo tanto
\begin{minted}{python}
print(x[len(x)-1])
\end{minted}
Imprime el 'ultimo caracter del string, cuyo 'indice es \texttt{len(x)-1}, debido que el 'indice del primer caracter es 0. El mismo resultado, puede ser conseguido usando
\begin{minted}{python}
print(x[-1])
\end{minted}
Similarmente,
\begin{minted}{python}
print(x[-2])
\end{minted}
imprime el pen'ultimo caracter, y as'i sucesivamente. Por ejemplo, ejecute
\begin{minted}{python}
x[-3]==x[len(x)-3]
\end{minted}
para comprobar que se refieren al mismo caracter.

\item Tambi'en es posible acceder a algunos caracteres del string usando, en nuestro caso, \texttt{x[inicio:fin:paso]}, donde \texttt{inicio} y \texttt{fin} los 'indices de los caracteres iniciales y finales y \texttt{paso} es un entero que define el paso. Si \texttt{paso} no es ingresado, se asume \texttt{paso=1} . Por ejemplo, ejecute y verifique qu'e hacen los siguientes comandos
\begin{minted}{python}
print(x[0:4:1])
print(x[0:4])
print(x[5:16])
print(x[1:20:2])
\end{minted}
Note que el caracter correspondiente al 'indice \texttt{fin} NO es desplegado. En lenguaje matem'atico podr'iamos decir que \texttt{x[inicio:fin]} suministra los caracteres de \texttt{x} con 'indices en el intervalo desde \texttt{inicio} \textit{cerrado} hasta \texttt{fin} \textit{abierto}.

\item Adem'as, si no se especifica \texttt{inicio} se asume el valor 0 (inicio del string) y si no se especifica \texttt{fin} se asume el valor \texttt{len(x)} (fin del string). Verifique esto ejecutando:
\begin{minted}{python}
print(x[:4])
print(x[5:])
print(x[5:-3])
print(x[::-1])
\end{minted}
\item \textbf{Bonus track}: 

?`Qu'e hacen los siguientes comandos?, ?`Modifican el valor de \texttt{x}?
\begin{minted}{python}
x.upper()
x.replace("a","e")
x.find("p")
x.find("semestre")
print(x)
\end{minted}
\item Otro concepto muy importante en Python es el de \textit{listas}. Las listas son  similares a las cadenas, excepto que cada elemento puede ser de un tipo diferente. La sintaxis para crear listas en Python es [..., ..., ...]. Por ejemplo, ejecute:
\begin{minted}{python}
lista = [1, "hola", 1.0, 1-1j, True]
type(lista)
print(lista)
\end{minted}
Como puede ver, la variable \texttt{l} es un nuevo tipo de objeto: 'list'. En este caso, es una lista cuyos elementos son un entero, un string, un float, un complejo, y un booleano. Para verificar esto, imprima el valor y el tipo de cada elemento de la lista. Por ejemplo,
\begin{minted}{python}
print(lista[0],type(lista[0]))
print(lista[1],type(lista[1]))
\end{minted}
Este ejemplo tambi'en muestra que los 'indices de cada elemento de la lista son numerados de la misma manera que en un string:
\begin{minted}{python}
print(lista[0:3])
print(lista[::2])
\end{minted}
\item Los elementos de una lista pueden tener cualquier tipo reconocido por Python, por ejemplo, pueden ser otra lista!:
\begin{minted}{python}
superlista=["cool",lista]
print(superlista)
\end{minted}
Imprima el valor y el tipo de cada elementos de esta lista. ?`Cu'antos elementos tiene la lista \texttt{superlista}? (respuesta, use la funci'on \texttt{len()}).

\item Existen diversas funciones en Python que crean listas. La funci'on \texttt{list()} crea una lista, por ejemplo, a partir de un string. Usando el string \texttt{x} definido anteriormente, ejecute
\begin{minted}{python}
y=list(x)
print(y)
print(type(y))
\end{minted}
\item Otra funci'on que crea listas 'utiles, esta vez de n'umeros \textit{enteros}, es \texttt{range(inicio,fin,paso)}, que crea una lista de valores desde \texttt{inicio} (cerrado) hasta \texttt{fin} (abierto!!), con paso \texttt{paso}. Ejecute,
\begin{minted}{python}
z=range(2,26,3)
print(z)
\end{minted}
Nuevamente, \texttt{(inicio,fin,paso)} funcionan de forma similar a los 'indices de un string o una lista:
\begin{minted}{python}
print(range(2,26))
print(range(26,2,-1))
\end{minted}
\end{questions}
\textbf{Bonus track}: 
?`Qu'e hacen los siguientes comandos?, ?`Modifican el valor de \texttt{x} y/o \texttt{lista}?
\begin{minted}{python}
x.split(" ")
x.split("e")
lista.append("chao")
lista.insert(2,"cool")
\end{minted}
\end{document} 
