\documentclass[11pt]{exam}
\usepackage[activeacute,spanish]{babel} % Permite el idioma espa\~nol.
\usepackage[latin1]{inputenc}
% !TEX encoding = UTF-8 Unicode
\usepackage{amsmath,amsfonts}
\usepackage[colorlinks]{hyperref}
\usepackage{graphicx}
\usepackage{hyperref}

%added because of problem between framed and exam package
\newcommand*{\renameenviron}[1]{%
  \expandafter\let\csname exam-#1\expandafter\endcsname
      \csname #1\endcsname
  \expandafter\let\csname endexam-#1\expandafter\endcsname
      \csname end#1\endcsname
  \expandafter\let\csname #1\endcsname\relax
  \expandafter\let\csname end#1\endcsname\relax
}
\renameenviron{framed}
\renameenviron{shaded}
\renameenviron{leftbar}
\usepackage{framed}
\usepackage{hyperref}


\pagestyle{headandfoot}
\spanishdecimal{.}

\begin{document}
\firstpageheadrule
%\firstpagefootrule
%\firstpagefooter{}{Pagina \thepage\ de \pages}{}
\runningheadrule
%\runningfootrule
\lhead{\bf\normalsize Taller Python 2019}
\rhead{\bf\normalsize Tarea \'Area Qu\'imica/Bioqu\'imica}
\cfoot{ }
\lfoot{\tiny EVM}
\begin{flushleft}
\vspace{0.2in}
%\hbox to \textwidth{Nombre: \enspace \hrulefill}
%Nombre : \\
\vspace{0.25cm}
\end{flushleft}
%%%%%%%%%%%%%%%%%%%%%%%%%%%%%%%%%%%%%%%%%%

\begin{center}
\textbf{Fecha de entrega: Lunes 21 de Enero}
\end{center}
\textbf{Instrucciones:} Resuelva el problema propuesto usando Python. Env'ie todos los archivos necesarios para reproducir sus resultados (archivos de datos, c'odigos .py, notebooks .ipynb, etc.) por email a \texttt{evohringer@udec.cl}.

\bigskip

En esta tarea usted deber\'a procesar la estructura cristalina de la prote\'ina hemoglobina obtenida a una resoluci\'on de 2.1 \AA\ mediante el m\'etodo de difracci\'on de rayos X. El archivo que contiene la estructura de la hemoglobina lo puede encontrar en \url{http://www.rcsb.org/pdb/ngl/ngl.do?pdbid=1GZX&bionumber=1} en donde debe descargar el archivo denominado PDB.

El formato del archivo es pdb "Protein Data Bank" que es el formato com\'un para estructuras de prote\'inas (Mayor informaci\'on puede obtener bajo \url{http://www.wwpdb.org/documentation/file-format-content/format33/v3.3.html}). 

El archivo contiene informaci\'on sobre la t\'ecnica de cristalizaci\'on, los detalles de la metodolog\'ia de Rayos X empleada etc. los cuales son especificadas como \textit{REMARK}, \textit{SEQRES} etc: 
\begin{verbatim}
REMARK   2                                                                      
REMARK   2 RESOLUTION.    2.1  ANGSTROMS.                                       
REMARK   3                                                                      
REMARK   3 REFINEMENT.                                                          
REMARK   3   PROGRAM     : PROLSQ                                               
REMARK   3   AUTHORS     : KONNERT,HENDRICKSON                                  
REMARK   3                                                                      
REMARK   3  DATA USED IN REFINEMENT.                                            
REMARK   3   RESOLUTION RANGE HIGH (ANGSTROMS) : 2.1                            
REMARK   3   RESOLUTION RANGE LOW  (ANGSTROMS) : 10                             
REMARK   3   DATA CUTOFF            (SIGMA(F)) : 0.0                            
REMARK   3   COMPLETENESS FOR RANGE        (%) : 95                             
REMARK   3   NUMBER OF REFLECTIONS             : 24327                          
REMARK   3                                                                      
REMARK   3  FIT TO DATA USED IN REFINEMENT.                                     
REMARK   3   CROSS-VALIDATION METHOD          : FREE R                          
REMARK   3   FREE R VALUE TEST SET SELECTION  : RANDOM                      
\end{verbatim}


Despu\'es de esta secci\'on se agregan lineas que contienen la informaci\'on de cada \'atomo en la estructura de la prote\'ina que comienzan con \textit{ATOM} o \textit{HETATOM} (cuando pertenecen a la cadena pept\'idica): 
\begin{verbatim}
ATOM      1  N   VAL A   1      18.432  -2.931   3.579  1.00 37.68           N  
ATOM      2  CA  VAL A   1      19.662  -2.549   2.806  1.00 35.41           C  
ATOM      3  C   VAL A   1      19.282  -1.939   1.441  1.00 34.04           C  
ATOM      4  O   VAL A   1      18.421  -2.497   0.695  1.00 33.95           O  
ATOM      5  CB  VAL A   1      20.659  -3.754   2.825  1.00 35.59           C  
ATOM      6  CG1 VAL A   1      20.109  -4.992   2.222  1.00 37.84           C  
ATOM      7  CG2 VAL A   1      21.982  -3.272   2.245  1.00 36.73           C  
ATOM      8  N   LEU A   2      19.905  -0.786   1.169  1.00 29.21           N  
ATOM      9  CA  LEU A   2      19.749  -0.064  -0.067  1.00 27.27           C  
ATOM     10  C   LEU A   2      20.513  -0.749  -1.213  1.00 27.19           C  
ATOM     11  O   LEU A   2      21.748  -0.901  -1.212  1.00 27.58           O  
ATOM     12  CB  LEU A   2      20.204   1.339   0.210  1.00 25.79           C  
ATOM     13  CG  LEU A   2      19.275   2.508   0.284  1.00 30.66           C 
\end{verbatim}
Estas l\'ieas contienen el \'indice del \'atomo, el tipo de \'atomo, el amino \'acido al cual pertenece y su n\'umero dentro de la secuencia, las coordenadas x, y, z de este \'atomo en Angstrom y al final el elemento.

\vskip 0.5cm

Escriba un programa que procese este archivo y entregue la siguiente informaci\'on:

\vskip 0.3cm


\begin{parts}
\part Una lista con los nombres de amino \'acidos que componen la cadena pept\'idica (secuencia) y un histograma con la frecuencia de ocurrencia de un amino \'acido en esa lista.


\part Las distancias entre cada uno de los cuatro \'atomos de hierro en \AA\ y ordene estas en una matriz de distancia (4x4) usando numpy.

\part La distancia entre los \'atomos de tipo \textbf{CA} del mismo amino \'acido presente en las cadenas A, B, C y D e identifique la distancia m\'axima, la m\'inima y el promedio.
\end{parts}
\end{document} 
