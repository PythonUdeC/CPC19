\documentclass[11pt]{exam}
%\usepackage[activeacute,spanish]{babel} % Permite el idioma espa\~nol.
\usepackage[utf8x]{inputenc}
%\usepackage[latin1]{inputenc}
\usepackage{amsmath,amsfonts}
\usepackage[colorlinks]{hyperref}
\usepackage{graphicx}

\pagestyle{headandfoot}

%\spanishdecimal{.}

\begin{document}

\firstpageheadrule
%\firstpagefootrule
%\firstpagefooter{}{Pagina \thepage\ de \pages}{}
\runningheadrule
%\runningfootrule
\lhead{\bf\normalsize Taller Python 2019}
\rhead{\bf\normalsize Tarea \'area Ingeniería}
\cfoot{ }
\lfoot{\tiny GR}
\begin{flushleft}
\vspace{0.2in}
%\hbox to \textwidth{Nombre: \enspace \hrulefill}
%Nombre : \\
\vspace{0.25cm}
\end{flushleft}
%%%%%%%%%%%%%%%%%%%%%%%%%%%%%%%%%%%%%%%%%%

\begin{center}
\textbf{Fecha Máxima de entrega: Lunes 21 de Enero}
\end{center}
\textbf{Instrucciones:} \noindent Resuelva el problema propuesto usando Python. Envíe todos los archivos necesarios para reproducir sus resultados (archivos de datos, códigos .py, notebooks .ipynb, etc.) por email a \texttt{svogt@udec.cl}.

\bigskip

\noindent El estudio de irradiación solar por zona geográfica es importante al momento de decidir si una región es adecuada 
para la instalación de celdas fotovoltáicas. En la carpeta de la tarea Ing-01 encontraran un archivo, \texttt{sun.csv} que contiene la información de 
irradiación solar sobre una grilla global de latitudes y longitudes. Esta información esta dada tanto por meses como anual. La base de datos 
entregada en la carpeta fue obtenida en \url{https://maps.nrel.gov/nsrdb-viewer} y procesada utilizando el script \texttt{pre$\_$proccess$\_$data.py},
para pasar de una grilla geográfica en polígonos a puntos.\\
\begin{parts}
    \part \noindent Grafique un histograma de la irradiación a nivel mundial y calcule el promedio global de irradiación. Despu\'es, construya 
    un histograma de irradiación tanto para el hemisferio norte como el hemisferio sur y presente las dos gráficas en una misma figura.
    \part \noindent La cuidad de Punta Arenas tiene coordenadas Lat/Long de (-53.163833, -70.917068) mientras que las coordenadas de Arica son:
    (-18.478253, -70.312599). Utilizando un longitud aproximada de -70  para todo Chile, construya una gráfica de latitud vs irradiación.
    A partir de la serie de puntos obtenida determine a qué latitud se encuentra el máximo de irradiación.
    \part \noindent Repita el análisis anterior para todos los meses del año e incluya las 12 gráficas en una misma figura.
    \part \noindent \textbf{Bonus track optional}:  Utilice las librerías \texttt{pyshp} o \texttt{arcpy} para el procesamiento de los datos en términos de polígonos, como aparece en la data cruda (\texttt{sun$\_$raw.csv}), y repita los ejercicios de la tarea. Si es que le interesan los análisis geográficos, le recomiendo familiarizarse con las librerías que fueron especialmente diseñadas para ese fin.
\end{parts}




\end{document} 
