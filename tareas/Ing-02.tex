\documentclass[11pt]{exam}
\usepackage[activeacute,spanish]{babel} % Permite el idioma espa\~nol.
\usepackage[utf8]{inputenc}
\usepackage{amsmath,amsfonts}
\usepackage[colorlinks]{hyperref}
\usepackage{graphicx}

\pagestyle{headandfoot}

\spanishdecimal{.}
%se redefine comando que cita figuras
\newcommand{\figref}[1]{\ref{#1}}
\usepackage{float}%para poder poner figuras justo en el lugar que deseo con \begin{figure}[H]
\usepackage{minted} 
\begin{document}

\firstpageheadrule
%\firstpagefootrule
%\firstpagefooter{}{Pagina \thepage\ de \pages}{}
\runningheadrule
%\runningfootrule
\lhead{\bf\normalsize Taller Python 2019}
\rhead{\bf\normalsize Tarea \'area Ingenier\'ia}
\cfoot{ }
\lfoot{\tiny NNIV}
\begin{flushleft}
\vspace{0.2in}
%\hbox to \textwidth{Nombre: \enspace \hrulefill}
%Nombre : \\
\vspace{0.25cm}
\end{flushleft}
%%%%%%%%%%%%%%%%%%%%%%%%%%%%%%%%%%%%%%%%%%

\begin{center}
\textbf{Fecha M'axima de entrega: Lunes 21 de Enero.}
\end{center}
\textbf{Instrucciones:} Resuelva el problema propuesto usando Python. Env'ie todos los archivos necesarios para reproducir sus resultados (archivos de datos, c'odigos .py, notebooks .ipynb, etc.) por email a \texttt{nibarra@ubiobio.cl}.

\bigskip
Sea $h(t)$ una funci\'on con $t \in \ ]-\infty,+\infty[$, se define la transformada de Fourier (FT por las siglas en ingl\'es de \textit{Fourier Transform}) continua ${\cal F} [h(t)](f):=\tilde{h}(f)$, como aquella transformaci\'on que aplicada sobre la funci\'on original, la descompone en una superposici\'on continua de funciones peri\'odicas con frecuencias $f$. Convencionalmente consideraremos la Transformada de Fourier ${\cal F}$ y su inversa ${\cal F}^{-1}$ como:
\begin{align}
\tilde{h}(f):=\int_{-\infty}^{\infty}h(t)e^{-i2\pi ft}dt, \qquad
h(t):=\int_{-\infty}^{\infty}\tilde{h}(f)e^{i2\pi ft}df.\label{eq:def-TF}
\end{align}

Cuando la se\~nal a estudiar corresponde a datos provenientes de un laboratorio o se han conseguido simulando alg\'un fen\'omeno f\'isico en el computador, no se tiene una expresi\'on anal\'itica a evaluar en la ec. \eqref{eq:def-TF}(a), sino que se tiene un conjunto de $N$ pares de n\'umeros $(t_{j}, h_{j})$, con $j=0,1,2,\dotsc,N-1$. Pese a ello, considerando la extensi\'on peri\'odica de la se\~nal muestreada durante un tiempo $T$ (tiempo total del muestreo), igual podemos definir la Transformada de Fourier Discreta (DFT por las siglas en ingl\'es de \textit{Discrete Fourier Transform}) como se indicar\'a a continuaci\'on.

Si la se\~nal est\'a uniformemente distribuida en el tiempo, tal que $t_j=j\Delta t$ con $j=0,1,\dotsc$ y $\Delta t$ el espaciamiento temporal, entonces se define la Transformada de Fourier discreta de la se\~nal y su inversa como 
\begin{align}\label{eq:def-UDFT}
\tilde{h}_k:=\frac{1}{N}\sum_{j=0}^{N-1}h_{j}e^{- i 2\pi k j/N},\qquad
h_j:= \sum_{k=0}^{N-1}\tilde{h}_{k}e^{i 2\pi k j/N}.
\end{align}

Note que la ec. \eqref{eq:def-UDFT}(a) se aproxima a la integral de Riemann de \eqref{eq:def-TF}(a), con $T=N\Delta t$ y $f_k:= k/T$.

\begin{enumerate}
\item Abra un Jupyter Notebook y copie el siguiente c\'odigo:

\begin{minted}{python}
import numpy as np #Se carga NumPy
import matplotlib.pyplot as plt #Se carga matplotlib.pyplot
import IPython.display as ipd #Se carga Ipython.display
from scipy.fftpack import fft, fftfreq #Carga Discrete Fourier Transform
from scipy.io import wavfile #Se carga modulo para leer y escribir archivos en formato wav

def fft_local(t,y):
    fourier = fft(y)
    freq = fftfreq(len(t), np.diff(t)[0])
    mask = np.where(freq>=0)[0]
    return [freq[mask], fourier[mask]]
    
tmax = 5. # Tiempo max. en segundos
fs = 22050 # Tasa de sampleo
t = np.linspace(-tmax,tmax,int(tmax*fs)) #arreglo de tiempo
dt = np.diff(t)[0]
\end{minted}

\item \label{p1} Con NumPy genere una se\~nal peri\'odica en la forma
\begin{align}
y_{1}(t) = A_{1}\cos(2\pi f_{1}t),
\end{align}
%
eligiendo arbitrariamente el valor de $A_{1}$ y $f_{1}$ (ambos positivos, se sugiere $A_{1}=1$ y $f_{1}=400$~Hz). Con la funci\'on \texttt{plt.plot} grafique $y_{1}(t)$ versus $t$ y con la funci\'on \texttt{ipd.Audio} resproduzca el sonido de la se\~nal generada pasando el p\'arametro \texttt{rate=fs} (tasa de sampleo).

\item \label{p2} Eval\'ue la DFT mediante la funci\'on \texttt{fff\_local} y grafique la magnitud de la transformada fourier versus la frecuencia.

\item Repita los puntos \ref{p1} y \ref{p2} para una se\~nal
\begin{align}
y_{2}(t) = A_{2}\cos(2\pi f_{2}t).
\end{align}

Se sugiere elegir $A_{2}=A_{1}$ y $f_{2}=2\,000$~Hz.

\item Repita los puntos \ref{p1} y \ref{p2} para la se\~nal
\begin{align}
y_{3}(t) = y_{1}(t)+y_{2}(t).
\end{align}

\item Desde \href{https://github.com/PythonUdeC/CPC19/blob/master/datas/CongaGroove-mono.wav}{aqu\'i} descargue el archivo de audio \texttt{CongaGroove-mono.wav} y c\'arguelo con la funci\'on \texttt{wavfile.read} copiando el siguiente c\'odigo:
\begin{minted}{python}
ffss, y4 = wavfile.read('./CongaGroove-mono.wav')
\end{minted}

Genere el arreglo de temporal asociado copiando el siguiente c\'odigo:
\begin{minted}{python}
dt4 = 1/fs4
N4 = len(y4) #numero de puntos
fN = 1./(2.*dt4) #frecuencia de Niquist
t4 = np.linspace(0, dt4*N4, N4) #arreglo temporal
\end{minted}

Con la funci\'on \texttt{plt.plot} grafique $y_{4}(t_{4})$ versus $t_{4}$ y con la funci\'on \texttt{ipd.Audio} resproduzca el sonido de la se\~nal cargada pasando el p\'arametro \texttt{rate=fs4} (tasa de sampleo). Repita el punto \ref{p2} para la se\~nal $y_{4}(t_{4})$.
 
\item Observe sus gr\'aficos, analízelos y en su correo responda: \textquestiondown Qu\'e informaci\'on de una se\~nal le entrega su respectiva DFT?
\end{enumerate}
\end{document} 
